\begin{abstract}
  % ~150 words
  % 1 Top-m arm identification comprehensible for any scientist
  Top-$m$ arm identification revolves around identifying the set of $m$ best out
  of $k$ options with rewards following unknown probability distributions.
  % 2 Top-m arm identification comprehensible for scientist from the field
  In particular, for a fixed budget of samples, one seeks to increase confidence
  in the estimate. Conversely, for a fixed confidence, one seeks to require the
  least amount of samples possible. Hence the optimization revolves around a
  fast convergence rate of the confidence for increasing samples.
  %1 The problem addressed by this paper
  Our work seeks to characterize this optimality concept more explicitly and
  intuitively.
  % 1 Summary of main result
  Defining the notion of evidence, we prove that the optimal allocation collects
  equal evidence to distinguish every pair of optimal and suboptimal arms.
  % 2 How the main result changes previous knowledge
  This characterization allows for a better intuitive understanding of the
  top-$m$ mechanism and should make it more convenient to prove convergence of
  algorithms. Moreover, we propose and analyze an algorithm which we conjecture
  to converge to the optimal allocation. Being very simple to understand and
  implement, our algorithm comes with confidence estimates and allows for
  leveraging prior knowledge.
  %!1 Result in general context
  %2 Broader perspective, understandable by any scientist
  In general, we intend this work to lay the foundation for top-$m$ convergence
  proofs, in particular for our suggested algorithm.
\end{abstract}
