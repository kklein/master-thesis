best arm\chapter{Background}\label{chapter:background}

\section{Bandits}
%What is the scenario/model?
The so-called stochastic multi-armed bandit is a general model to simulate
decision-making in uncertain environments. In particular, one assumes a set of
arms to choose from, each choice leading to an outcome, referred
to as reward. In general, one assumes many sequential selections among the set
of arms. The outcome of the selection of an individual arm typically follows a
fixed but unknown probability distribution.
%What problems are typically tackled by this scenario?
Within the bandits model, the concern revolves around which arm to choose next.
Allocation strategies, also referred to as measurement plans, tackling this
question address either of two problems: explore-exploit or pure-explore.

More concretely, the former concerns itself with maximizing the \emph{cumulative
reward}. This means that one attempts to maximize the \emph{sum of the rewards
obtained} through all arm selections. Naturally, starting off without any
knowledge about the underlying distributions, this involves both exploration of
arm qualities as well as exploitation of knowledge obtained so far.

The latter revolves around \emph{simple reward}. This problem consists of
seeking to maximize the reward one obtains if one leverages the current
knowledge for \emph{another} draw. This implies that the focus lies on
\emph{identification} of high-quality candidates, quality usually being assessed
by high means. The main task is therefore to estimate the true best arms
with high \emph{confidence}.

%Why is model and its problems relevant?
The bandits model can be used for all processes involving sequential decision
making. Yet, an important simplification is the assumption of the distributions
being fixed over time. To what extent this simplification is representative of
the to-be-modeled process varies.

The explore-exploit problem can be found in Recommender Systems \cite{McInerney:2018:EEE:3240323.3240354}, ad campaigns \cite{5360225},
in the context of Reinforcement Learning
\cite{DBLP:journals/corr/abs-1811-12560} \cite{Szepesvari:2010:ARL:1855083} and
Robotics \cite{Baldassano}.

For real-world applications of the pure-exploration bandit please refer to
\ref{ss:top-1_model}.

\section{Notation and definitions}\label{section:notation}
We assume $k$ arms to choose from and denote the set of possible arms as $[k]$.
Subsets $S \subset [k]$ are assumed to be of size $m < k$.

A possible set of means for the arms is denoted as the $k$-dimensional vector
$\theta$, where every mean can lie between 0 and 1. We also refer to such a
$\theta$ as parameter.

When referring to an individual arm in the top-$m$ case, we proceed to use $j$
for arms in $S^*$, $i$ for arms not in $S^*$ and $l$ for arms of which this
knowledge does not exist. Finding ourselves in a frequentist setting, we assume
an underlying true mean of the arms. We refer to this as $\theta^*$. Moreover,
this ground truth implies a true best arm $l^*$ and true top-$m$ arms $S^*$.

Given the constraint that the means lie between 0 and 1, there are infinitely
many possible parameter vectors which make arm $l \in [k]$ the best one under
$\theta$ or $S \subset [k]$ top-$m$ under $\theta$. Hence we group such $\theta$
in the following way:
\begin{align}
  \Theta &:= [0, 1]^k \\
  \Theta_{1, l} :&= \{\theta \in \Theta | l = \argmax_{l' \in [k]}
      \theta_{l'}\} \\
  \Theta_{m, l} :&= \{\theta \in \Theta | l \in \text{ top-}m(\theta)\} \\
  \Theta_S :&= \{\theta \in \Theta | S = \text{top-}m(\theta)\} \\
    &= \{ \theta \in \Theta| \min_{j_1 \in S}\ \theta_{j_1} > \max_{j_2 \notin
      S}\ \theta_{j_2}\}
\end{align}
where top-$m$ returns the $m$ highest values, i.e.\ means, from its argument
$\theta$.

After having made $n$ observations of rewards, bundled in $D_n$, $\Pi_n$
expresses a posterior distribution with density $\pi_n$ over elements of
$\Theta$. e.g.\ for sets $\Theta_{S}$, we have:
\begin{align}
  \Pi_n(\Theta_{S}) &:= \int_{\theta \in \Theta_{S}} \pi_n(\theta)
      d\theta \\
    &= \Pr[S \text{ is top-}m | D_n]
\end{align}
Clearly, it holds that $\Pi_n(\Theta) = 1$. As a shorthand for the top-$m$ case, we will use:
\begin{align}
  \alpha_{n, S} &:= \Pi_n(\Theta_{S}) \\
  \alpha_{n, l} &:= \sum_{S: l \in S} \Pi_n(\Theta_{S}) \\
\end{align}
We are mostly interested in strategies that allocate measurement effort over the
arms in a randomized fashion, i.e.\ according to a probability distribution. We
refer to such an allocation or measurement plan as $\psi \in [0, 1]^k$, defining
the probability of each arm being sampled. Naturally, it holds that $\sum_{l \in
[k]} \psi_l = 1$. We also define the allocation property of a set by the sum of
its arms: $\psi_S = \sum_{l \in S} \psi_l$.

The $n$-th sampled arm is denoted as $l_n$. For adaptive strategies, the
allocation can change after every sample. After $n$ samples we refer to the
allocation as $\psi_{n}$ and therefore we have $\psi_{n, l} = \Pr[l_n = l]$ for
arm $l$. We might Also be interested in the average allocation up to a
certain sample $n$. We write:
\begin{align}
  \bar{\psi}_{n, l} &:= \frac{\sum_{n' = 1}^{n} \psi_{n', l}}{n}
\end{align}
We use the notation $d(\theta_1||\theta_2)$ to represent the Kullback-Leibler
divergence between $\theta_1$ and $\theta_2$.

We employ Russo's notation $a_n \deq b_n$ if the the following relationship
holds:
\begin{align}
  \lim_{n \rightarrow \infty}\frac{1}{n}\log{\frac{a_n}{b_n}} \rightarrow 0
\end{align}
Hence $\deq$ can be vaguely thought of as asymptotic relationship indicating
similar growth. Note the asymmetry of the relationship: it rather
represents an inequality than an equality. Russo points refers to this relation as \emph{logarithmic equivalence} and points out that $a_n$ and $b_n$ are \emph{equal up to first order in the exponent}. Clearly, it holds that $a_n + b_n \deq \max\{a_n, b_n\}$ and $c a_n \deq a_n$ for positive constant $c$.

\section{Thompson sampling for bandits}
%How does it work?
%Why is it useful?
%What is it used for?

Thompson sampling is a sampling strategy often employed for selecting arms in
the bandits scenario. By default, it particularly lends itself to the cumulative
reward setting. In the following, we will discuss Thompson sampling in the
context of bandits.

The foundation of Thompson sampling is a Bayesian approach: instead of 'only'
possessing estimates on the means/distributions of arms, it entails a
distribution \emph{over} arms, indicating the likelihood of a random variable
over the arms. In the bandits scenario said random variable tends to be the mean
of an arm. Hence, instead of only point estimates for per arm, Thompson sampling requires a distribution over the arms.

Naturally, this distribution over the arms should also leverage the knowledge
acquired throughout the sampling process. Hence, per round, we can talk about
\emph{prior} and \emph{posterior} distributions. This posterior distribution of
a step $n$ answers the question: 'How likely are the arms $1, 2, 3$ to have
means $[.193, .254, .1]$ after having observed samples 1 to $n$?'. As this is an
iterative process, the posterior after observing sample $n$ will serve as the
prior for selecting sample $n+1$. Hence the posterior is the the prior
\emph{updated} with the knowledge acquired of a specific round. Thanks to this
duality we will continue to only talk about posteriors.

One might wonder how exactly posteriors are updated. The update procedure is
dependent of the type of utilized class of distribution for prior and
posteriors. An example of this can be found in \ref{section:empirical_behavior}.

In contrast to greedily selecting the arm that empirically maximizes the metric
of desire, e.g.\ the mean, Thompson sampling suggests to draw from posterior.
Note that said posterior distribution is itself a function of the empirical
means. Subsequently, it selects the arm which maximizes the metric of desire on
the sampled belief. With many repetitions, this leads to each arm being sampled
proportionally to its likelihood of maximizing the metric of desire according to
current belief. \Cref{alg:thompson_sampling} illustrates this process more
explicitly.

\begin{algorithm}[H]
    \caption{Thomson sampling: Given a posterior $\Pi_{n-1}$ in step $n$}
    \label{alg:thompson_sampling}
  \begin{algorithmic}
    \State $\hat{\theta} \sim \Pi_{n-1}$
    \State $l_n = \argmax_{l \in [k]}\hat{\theta}_l$
    \State Play $l_n$
    \State Observe reward $r_n$
    \State $\Pi_n = update(\Pi_{n-1}, l_{n+1}, r_{n+1})$
  \end{algorithmic}
\end{algorithm}

Intuitively, this is very appealing for the cumulative reward scenario as it
creates a natural balance between exploration and exploitation.

\section{Best arm identification}
\subsection{Model}\label{ss:top-1_model}
%What is the problem?
%Why is it relevant?
%How is performance evaluated? What is 'confidence'?

Best arm identification implies disregarding the sum of the rewards encountered
while sampling. Instead, the goal is to efficiently gather information
maximizing the confidence of the suggestion made \emph{after} the sampling
phase. In other words, there is a purely explorative phase, also referred to as
'experiment' which is then typically followed by a purely exploitative phase,
having committed to an option. For this reason, best arm identification is
closely related to Optimal Experiment Design.

Quite naturally, all other aspects being controlled for, requiring fewer samples
is preferable. Hence one can formulate the goal in two ways: maximize confidence
for a given amount of samples or minimize amount of samples for a given
confidence level.

Up to this point we only talked about confidence in vague terms. As a matter of
fact, different approaches for best arm identification rely on different
approaches to express their confidence in the estimation. Herein lies a
substantial differentiation and therefore, naturally, it is not trivial to
compare different approaches other than by investigating what their best arm
output is, i.e.\ disregarding the confidence. For Bayesian algorithms, it is
possible to quantify the confidence the model has in the currently most
favorable looking candidate. This quantity corresponds to the mass a posterior
puts on parameters favoring said candidate. Another approach to measure
confidence stems from the realm of Probably Approximately Correct learning, or
PAC in short. In the PAC context, one desires to make a statement of the sort
$\Pr[\text{output of algorithm is $\epsilon$-correct}] \geq 1 - \delta$, where
$\epsilon$-correctness requires an explicit definition. In this case, under the
toleration of $\epsilon$, $1 - \delta$ can be thought of as confidence.

Real-world applications of best arm identification include:
\begin{itemize}
  \item Physical simulations: Given a collection of designs, e.g.\ the bodywork
  of a car, determine aerodynamic properties of the designs. Elaborate
  simulations can come with significant resource demands, such as compute power
  as well as the direct and indirect costs of time. Arriving at the same
  conclusion of the superiority of a certain designs, with fewer simulations can
  be extremely valuable.
  \item Crop selection: Experimenting different crop types in a given growing
  environment measuring yields, generate a recommendation for that same growing
  environment.
\end{itemize}


\subsection{Optimal fixed allocation}\label{subsection:optimal_allocation}
% What is the overall goal?
% How is this achieved? (Talk about optimization over hyper-parameters)

We start off by describing what it means for an allocation to be optimal and by
enumerating some of its properties. We do not present a constructive allocation,
rather we assume knowledge about the underlying truth to provide a tight bound
on the best possible fixed allocation. Intuitively it is not possible to match
the performance of that optimal allocation without the knowledge of the
underlying truth. Hence a very desirable statement is to show that a proposed,
constructive adaptive allocation \emph{converges} to the optimal allocation.

As mentioned in \ref{ss:top-1_model}, the goal consists of maximizing
confidence, i.e.\ the mass the posterior lays onto the true best arm. Observe
that for any allocation sampling each arm with at least some probability, the
confidence will tend towards 1. What makes the optimal allocation optimal is the
rate at which the posterior of the true arm converges towards 1.

Note that the optimal allocation assumes knowledge about the underlying true
value and therefore does not need to be adaptive. It can be framed as the
following thought experiment: Assume you know the true underlying means. Yet
your adversary doesn't trust your 'knowledge'. He only trusts the sample rewards
that he can observe for himself. Now it is your task to leverage your knowledge
about the true means to sample arms in a fashion convincing the adversary as
quickly as possible.

\subsubsection{Rate of convergence}
Recall that $\Theta_{1, l^*}$ is the set of parameters under which the true best
arm $l^*$ is optimal. Russo \cite{DBLP:journals/corr/Russo16} shows that the
rate at which the posterior $\Pi_n$ of $\Theta_{1, l^*}$ converges towards 1
cannot be faster than the following:
\begin{align}
  \Pi_n(\Theta_{1, l^*}) &= 1 - \Pi_n(\Theta^c_{l^*}) = 1 - \exp\{-n\Gamma^*\} \\
  \Gamma^* &= \max_{\psi} \min_{\theta \in \Theta^c_{l^*}} \sum_{l \in [k]}
      \psi_l d(\theta_l^* || \theta_l)
\end{align}
where $n$ corresponds to the number of samples acquired.

The allocation $\psi$ maximizing this quantity is what we refer to as optimal
allocation.

\subsubsection{Defining properties}
Russo's underlying idea is that the optimal allocation gathers equal
\emph{evidence}, e.g.\ compared to having equal measurement effort, as  for a
uniform distribution. This notion of evidence relies on the comparison between
the true best against all other arms. It takes both their respective true means
as well as sampling frequencies into consideration. He defines
\begin{align}
  C_i(p_1, p_2) := \min_{x \in \mathbb{R}} p_1 d(\theta_{l^*}^*||x) + p_2
      d(\theta_{i}^*||x)
\end{align}
Where $p_1$ and $p_2$ will typically correspond to the measurement allocations
$\psi_{l^*}$ and $\psi_{i}$:
\begin{align}
  C_i(\psi_{l^*}, \psi_i) := \min_{x \in \mathbb{R}} \psi_{l^*}
      d(\theta_{l^*}^*||x) + \psi_i d(\theta_{i}^*||x)
\end{align}
He goes on to show that the optimal allocation $\psi^*$ is
identified by fulfilling the condition
\begin{align}
  \forall i_1, i_2 \neq l^*: C_{i_1}(\psi_{l^*}, \psi_{i_1}) =
      C_{i_2}(\psi_{l^*}, \psi_{i_2})
\end{align}
Intuitively, this expresses that for every suboptimal arm, an equal amount of
evidence of suboptimality has been gathered.

Moreover, Russo goes on to show the uniqueness of the optimal allocation.

\subsection{A constrained optimal allocation}
%What are properties of the constrained optimal allocation?
%How can they be interpreted?

In order to bridge the gap between algorithm and optimal allocation, Russo
introduces the concept of \emph{constraining} the optimal allocation. His
algorithm naturally implies the constraint that in the limit, $\beta$ of the
measurement effort is allocated to the true best arm.

He is able to show that his algorithm's allocation converges to the overall
optimal allocation by showing that
\begin{itemize}
  \item The algorithm's average allocation $\bar{\psi}_n$ converges to the
  optimal allocation under the constraint $\psi_{l^*}^* = \beta$
  \item The hyperparameter $\beta$ can be tuned to equal the overall optimal
  value.
\end{itemize}
The optimal convergence exponent under said constraint becomes:
\begin{align}
  \Gamma^*_{\beta} &= \max_{\psi: \psi_{l^*} = \beta} \min_{\theta \in
      \Theta^c_{l^*}} \sum_{l \in [k]} \psi_l d(\theta_l^* || \theta_l)
\end{align}

\subsection{Top-Two Thompson sampling algorithm}
% What is the algorithm?

Russo proposes different algorithms that satisfy the aforementioned theoretical
results, yet suggests that one of them outperforms the other empirically:
Top-Two Thompson sampling (TTTS) algorithm.

The main idea behind TTTS is to repeat obtaining candidates through Thompson
sampling until two different candidates have been proposed. This illustrates how
Thompson sampling, most commonly used for exploit-explore settings can be used
for pure-exploit: some of its focus is shifted towards inferior-looking
candidates. Among those two candidates, the first to have been drawn is picked
with probability $\beta$, explaining the provenance of the hyperparameter. Note
the difference between Thompson sampling, in \Cref{alg:thompson_sampling} and
TTTS, in \Cref{alg:TTTS}: an additional level of 'randomization'.

\begin{algorithm}[H]
    \caption{TTTS: Given a posterior $\Pi_{n-1}$ in step $n$}
    \label{alg:TTTS}
  \begin{algorithmic}
    \State $\hat{\theta} \sim \Pi_{n-1}$
    \State $l_1 := \argmax(\hat{\theta})$
    \Repeat
      \State $\hat{\theta} \sim \Pi_{n-1}$
      \State $l_2 := \argmax(\hat{\theta})$
    \Until{$l_1 \neq l_2$}
    \State $B \sim Bernoulli(\beta)$
    \If{$B=1$}
      \State $l_n := l_1$
    \Else
      \State $l_n := l_2$
    \EndIf
    \State Play $l_n$, observe reward and update priors
  \end{algorithmic}
\end{algorithm}

Russo's formal treatment of this algorithm relies on the assumption that
observations follow 1-dimensional distributions belonging to the family of
exponential distributions. Moreover, he allows priors that are non-conjugate to
the posteriors, though does not discuss how to update or sample from them.

\section{Top-$m$ arm identification}
% What is the problem?
% Why is it relevant compared to Top-1?

Top-$m$ arm identification is a generalization of best arm identification. The
objective is to identify the set of arms, with cardinality $m$, which contains
the $m$ best arms.

Applications of top-$m$ arm identification are similar to those mentioned in
\ref{ss:top-1_model}. Natural motivations for the identification $m$
instead of a single high-quality option can be diversification or regulation.

\subsection{A common approach: LUCB}
The general Lower Upper Confidence Bound algorithm is a common approach for
top-$m$ arm identification \cite{kaufmann2013information} and typically evaluated in the PAC setting. It can be
seen in \Cref{alg:LUCB}.

For every arm $l$, an empirical mean $\hat{\mu}_l$ is kept and updated after
each sample. The overall idea is to compute a confidence bound for every arm, in
every step. Per round $t$, arms are separated into two sets: the arms with the
$m$ best empirical means, $Top(t)$ and the rest, $Bottom(t)$. The arm with the
lowest lower confidence bound from $Top(t)$ and the arm with the hightest upper
confidence bound from $Bottom(t)$ are sampled and all information updated. We
refer to those arms as $tl$ and $bu$ respectively. The algorithm stops once its
stopping criterion is met. The confidence bound of an arm $l$ in a given step
equals $(\hat{\mu}_l - \beta(l), \hat{\mu}_l + \beta(l))$. Kalyanakrishnan et
al. \cite{DBLP:conf/icml/KalyanakrishnanTAS12} propose a concrete instantiation
by defining the estimation of the confidence bound and stopping crierion:
\begin{align}
  \beta(l) &= \sqrt{\frac{1}{2c_l}\ln\frac{knt^4}{\delta}} \\
  \xi &= (\hat{\mu}_{bu} + \beta(bu) - (\hat{\mu}_{tl} - \beta(tl))
\end{align}
with constant $k=\frac{5}{4}$, $c_l$ the number of times arm $l$ has been
sampled and $t$ the number of steps completed so far.
\begin{algorithm}[H]
    \caption{LUCB: Given prior means}
    \label{alg:LUCB}
  \begin{algorithmic}
    \State Sample each arm once
    \State Update $\hat{\mu}$ and confidence bounds
    \Repeat
      \State Compute $Top(t), Bottom(t), tl, bu$
      \State Sample $tl$ and $bu$
      \State Update $\hat{\mu}$ and confidence bounds
    \Until{$\xi < \epsilon$}
  \end{algorithmic}
\end{algorithm}
