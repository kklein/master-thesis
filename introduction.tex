% Some commands used in this file
\newcommand{\package}{\emph}

\chapter{Introduction}

% \begin{itemize}
%   \item Algorithm and characterization for top-m arm identification as a generalization of Russo's top-1 work.
%   \item What is the relationship between the optimal allocation and the algorithm?
%   \item Talk about optimal (fixed) vs adaptive.
%   $\rightarrow$ Mention the the outlook of proving convergence based on results
%   we've established.
%   \item Algorithm details/constraints:
%   \begin{itemize}
%     \item What are constraints on prior and posterior distributions?
%     \item Are we in a frequentist or Bayesian setting?
%     \item What is particular about the algorithm being Bayesian?
%   \end{itemize}
%   \item How do both statements translate to real world applications?
%   \item Explain the flow of the document.
% \end{itemize}

Much of Machine Learning revolves around how to learn from observations.
Technological advances allow for an extensive gathering of data in a plurality
of domains. Notwithstanding, many domains are still, and likely to remain, areas
of expensive data acquisition. Due to immense complexity, human behavior as well
as many physical processes can only be simulated at very high costs or not at
all. As a consequence, data is gathered by experiments that can be long-lasting,
strictly constrained in quantity and precious. In light of that, we focus on the
following aspect of Machine Learning: how to acquire observations.

In order to capture the complexity of real-world processes, we model outcomes of
decisions according to probability distributions. More precisely, we make use of
the well-established multi-armed bandits model for that exact purpose. As we've
mentioned, we seek to describe a manner of acquiring, or sampling, observations.
Naturally, one must ask the question: 'With what exact goal?'

We assume the goal of the data generation and acquisition to be the
identification of the $m$ best out of $k$ options. Within the realm of this
goal, we attempt to tackle the following problems:
\begin{itemize}
  \item What is, in theory, the best possible acquisition strategy?
  \item What is an algorithm that can, under some constraints, mimic this best possible acquisition strategy?
\end{itemize}
In particular, we seek to address these challenges by generalizing work from
Russo \cite{DBLP:journals/corr/Russo16} on identification of the single best
option. Consistent with existing literature, we will continue to refer to
options as arms. This terminology stems from the image of a many-armed bandit
slot machine.

We assume a frequentist setting. This implies that the options follow fixed
distributions, which are unknown to us. Yet, we can learn about those unknown
distributions by observing the samples the environment draws. This can be
thought of as being confronted with a noise-free signal that is then polluted
with noise from a sensor. In this spirit, we seek to identify the true best
options with as few samples as possible and be as confident of our
recommendation as possible, which will be formalized.

Our first contribution is the characterization of the optimal acquisition
strategy, also referred to as optimal measurement plan or allocation. What makes
the optimal allocation optimal is the rate of increase in confidence in the true
best options per sample. This optimal strategy is based on a thought experiment
of knowing the underlying truth and acquiring samples validating the knowledge
as well as possible. As a consequence, it serves as a bound for practical
algorithms outside of thought experiments. Intuitively, the guiding theme of the
characterization is that the allocation seeks to gather equal \emph{evidence} on
every option. We will provide a definition and intuition of evidence closely
following Russo's work.

Our second contribution is the definition of a concrete and simple adaptive
algorithm. As compared to the optimal allocation, the algorithm changes its
measurement plan after every sample, based on the observations it has made. The
algorithm, being Bayesian, comes with the upside of confidence estimation and
allows for leveraging and expressing domain knowledge through the use of priors.
We analyze properties of the algorithm and highlight insights that we deem very
useful for proving asymptotic convergence of this algorithm's allocation to the
optimal allocation.

In those undertakings we make very few assumptions on the underlying conditions.
Concretely, we expect the options to follow distributions from the
one-dimensional canonical exponential family. This includes common distributions
such as the Bernoulli, binomial with known number of trials, Poisson,
exponential, Pareto with known minimal value, chi-squared or the normal
distribution with known
variance \footnote{\url{https://en.wikipedia.org/wiki/Exponential_family}}.

\Cref{chapter:background} introduces the general problem context as well as
Russo's work, which we attempt to generalize from top-1 to top-$m$ selection. We
characterize the optimal allocation in \Cref{chapter:optimal}. Our proposed
algorithm, Top-2$m$ XOR Thompson sampling, is presented and analyzed in
\Cref{chapter:algorithm} before concluding our work in
\Cref{chapter:conclusion}.
