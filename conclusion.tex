\chapter{Conclusion}\label{chapter:conclusion}

In \Cref{chapter:optimal} we introduced a notion of evidence for the top-$m$ arm
identification case and characterized a fixed optimal allocation through it.
This allows for the computation of a best-possible convergence rate,
establishing intuition on the top-$m$ optimization mechanism and computation of
concrete optimal allocations as benchmarks for cases in which the underlying
truth is at hand.

In \Cref{chapter:algorithm} we presented our simple, adaptive Bayesian algorithm
for top-$m$ arm identification. Relying on Thompson sampling and an additional
layer of randomization, we presented theoretical results regarding its
measurement allocation. In addition, provided intuition as to why we expect it
to sample the most relevant arms in a given step. Furthermore, we've proven
statements that we conjecture to be very useful to show that this algorithm
converges to the optimal allocation characterized in \Cref{chapter:optimal}.
Qualitative empirical results from simulations have not contradicted our
statements or intuitions.

Proving that TXTS converges to the fixed optimal constrained allocation seems to
require some more elaborate techniques than employed by Russo for his top-1
case. Nevertheless, thanks to the empirical results, the nature of the algorithm
being a natural generalization of Russo's top-1 algorithm and the already
established theoretical foundations, we suggest the conjecture holds.
